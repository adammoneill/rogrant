% !TEX root = main.tex
\section{Introduction}

The main goal of the proposal is to study various influential schemes proven secure in the random oracle (RO) model and identify plausible standard (RO devoid) properties of the oracles and constituent primitives (such as base encryption schemes or trapdoor permutations) that suffice to prove security. To assess plausibility of the assumptions, we also seek theoretical constructions meeting them.  The proposed work thus increases our confidence in the schemes, helps us better understand their security, and explains why they have stood up to years of cryptanalysis despite the fact that the RO model is unsound in general.

\subsection{Background and Motivation}

In the random oracle (RO) model of Bellare and Rogaway~\cite{CCS:BelRog93}, every algorithm has oracle access to the same truly random functions.  It has been enormously impactful in enabling the design of practical protocols for various goals; examples include public-key encryption~\cite{CCS:BelRog93,EC:BelRog94}, digital signatures~\cite{CCS:BelRog93,EC:BelRog96}, and identity-based encryption~\cite{BonFra03}. In practice, one ``instantiates'' the RO oracles, that is, replaces their invocations with invocations of functions with publicly-avaiable code, resulting in a particular ``instaniated'' version of protocol. The idea is that in practice such functions could be appropriately built out of cryptographic hash functions.  The \emph{RO model thesis} is that the instantiated protocol remains secure in the standard (RO devoid) sense.

The RO model is often thought of as simply a heuristic.  However, note that a security proof is always relative to a model that abstracts away some details of the system. For example, the standard model abstracts away side-effects of physical computation~\cite{TCC:MicRey04}.  In fact, a proof in the RO model \emph{does} guarantee absence of   attacks treating the  functions as a black-boxes.  But it does \emph{not} rule out successful attacks taking advantage of their code.  This has been exploited in many works demonstrating complete failure of the above thesis, starting with that of Canetti \emph{et al.}~\cite{JACM:CanGolHal04}.  That is, these works construct RO model schemes for which \emph{any} instantiation yields a scheme that can be attacked. 

However, such schemes or their goals are all contrived or artifical in some way.  Indeed, RO model schemes that have been standardized and implemented have stood up to tens of years of cryptanalysis.  What explains this?  This leads to what may be called the \emph{practical RO model thesis:} For a ``practical'' scheme, instantiating it  appropriately via  a cryptographic hash function results in a secure scheme in the standard model.  However, from a scientific standpoint this thesis is usatisfactory because it lacks a \emph{definition} of  ``practical.''\footnote{We are reminded of the US Supreme Court's ruling in the case of Jacobellis v.~Ohio~\cite{wiki:JO}, in which Justice Potter Stewart's opinion did not define pornography but instead claimed ``I know it when I see it.''}  

I observe that another way of looking at the above is to ask what \emph{makes} schemes like RSA-OAEP~\cite{EC:BelRog94} or RSA-PSS~\cite{EC:BelRog96} ``practical'' and fall into a class of schemes to which the RO thesis applies.  This leads to a functional definition of ``practical'' and a reformulated practical RO model thesis:

\begin{quote} \label{ques1}\emph{
For ``practical'' RO model schemes, there exist plausible standard model properties of the constituent functions that suffice to prove security. It is to these schemes that the RO model thesis applies.}
\end{quote}

According to this thesis, after one proposes what one believes to be a ``practical'' RO model scheme to which the RO model thesis applies, one should search for such standard model properties to justify this belief.  It is furthermore desirable to back up plausiblity of an assumption with a theoretical construction meeting it, but we believe this is not strictly necessary as long as the assumption cannot be disproved. 	Notably, while impossiblity results in the standard model are known for RSA-OAEP~\cite{EC:KilPie09} and RSA-PSS~\cite{C:DodOliPie05,TCC:DodHaiTen12}, these are \emph{black-box} impossiblity results that demonstrate that a \emph{reduction} treating the functions as a black-box cannot suffice to prove security of these schemes in the standard model.  As in other areas of cryptography~\cite{FOCS:Barak01}, this motivates looking at non-blackbox assumptions.  I also emphasize that such proofs can moreover exploit novel assumptions on the constituent primitives, like RSA or ElGamal.

One approach to validating this ``functional'' version of the practical RO model thesis is to develop novel standard model notions that suffice to instantiate a wide variety of RO model schemes. However, I take an orthogonal approach in which I study  specific  schemes and try to justify they belong to the above class.  The reason is that notions that suffice to instantiate a wide variety of RO model schemes are often quite strong and tricky in that certain forms of them may be of them may be provably impossible in general.  Examples of such primitives are correlation intractable hash functions~\cite{JACM:CanHalGol04}, and universal computational extractors~\cite{C:BelHoaKee13,C:BelHoaKee14}.  Certain forms of these notions are provably impossible in general~\cite{JACM:CanHalGol04, AC:BrzMit14} but are possible (under strong assumptions) in special cases~\cite{EPRINT:BrzMit15,TCC:CanCheRey16,EC:CCRR18}.  In some sense, this just reduces finding when instantiating ROs is possible to finding when these notions are possible.  On the other hand, encouraging results like my work showing IND-CPA security of RSA-OAEP~\cite{EC:BelRog} use relatively simple assumptions. 

\subsection{Setting and Goal} 

  My proposal is based on three subgoals:
 \begin{itemize}
 \item X
 \item X
 \item X
  \end{itemize} 

 
  
 \subsection{Intellectual Merit and Broader Impact}
 
  \paragraph*{Intellectual Merit.}  
 \paragraph*{Broader Impact.} \label{sec-outreach}
 
 The proposed research will help build  a more secure Internet and digital infrastructure,  enable  new economies, and allow government, healthcare, and military organizations to take advantage of emerging  computing environments   in a responsible and privacy-preserving way, which are clear national priorities.  
%Additionally,  the proposed research will provide foundations for work done in the database and networking communities.
% The results will be widely disseminated via
%papers at the appropriate academic venues. %
%We  also hope to develop prototype implementations of our proposed  schemes and oversee their standardization and deployment.
To ensure these broader impacts, I have developed an integrated outreach plan consisting of the following: %that leverages the unique opportunities afforded by Georgetown's location in Washington DC and  its  strengths in policy and medicine.  

\begin{itemize}
\item Working with Deputy Cybersecurity Director Donna F. Dodson, I  obtained a guest faculty position in NIST's Cryptographic Technology Group.
One of their current focuses is ``lightweight cryptography,'' which is  connected to this proposal in that the ODB protocols I develop could be potentially run using commodity devices like cell phones by the client.  I will work with and disseminate my results widely at NIST.

\item I am planning to offer a course in cryptography and information security to female eighth graders at the ``Eureka!,'' a program affiliated with UMass Amherst that teaches STEM to female under-represented minorities. (\emph{Note I am moving from Georgetown University to UMass Amherst in January 2019.}) More generally, I think cryptography should be pushed further down the curriculum chain to not just undergrads but high school students.  This idea is related to the ``CS for All'' initiative.  In a sense, I want a ``Cryptography for All'' movement where high school kids understand end-to-end encryption and Facebook's ability to see their data.  I am committed to contacting local high schools to arrange lectures and summer internships.

\item I am very interested in policy  issues in cryptography, \emph{e.g.} the recently proposed US bill that would disallow the government from requiring a backdoor to end-to-end encryption apps, and have reached out to the School of Public Policy and the chair of political science at UMass, Jane Fountain,  about potentially offering a joint course with both Computer Science and Policy students and collaborating on research in crypto policy.  %(See my attached Letter of Collaboration from Jane Fountain, Department Chair of  Political Science and Public Policy at UMass.)
\end{itemize}

Finally, I believe an important aspect of my work is collaboration with applied researchers, which I have a track record of~\cite{SIGMOD:MCOKC15,CCS:KKNO16,KKNO17,ZOSZ17}.

%%%%%%%% JUNKYARD

\iffalse

\paragraph*{Attacks Considered.} Current work on ODB protocols generally considers ``snapshot'' and ``persistent'' attacks.  In a snapshot attack, the adversary gets one static view of the ``encrypted'' DB without any knowledge of queries and their responses and changes to the DB over time (if updates are supported). In a persistent attack, the adversary also sees the latter.  It's hard to map these types of attacks precisely onto the security threats mentioned at the beginning of the proposal, but roughly speaking the first corresponds to neither (motivating the consideration of other attacks), the third maps to a snapshot attack, and the second and fourth to a persistent attack.        Interestingly, Grubbs~\emph{et al.}~\cite{Grubbs:2017} argue that  any reasonable adversary that could perform a snapshot attack via the third threat would also get access to some information about the queries and responses via caches and logs, motivating the need to consider ``intermediate'' attacks.

\paragraph*{Current Attacks and Meaningful Notions} 
Note that the above cannot be made into any rigorous \emph{security notion} for ODB protocols in a provable-security sense, because they talk of the adversary's \emph{capabilities} but not its \emph{goal}.  Indeed, initial work on ODB protocols stipulated the adversary could compute no more than some specific ``leakage function'' of the database, but this does not translate to any ``meaningful'' security guarantee for a client.  
Indeed, ``inference'' attacks emerged on ODB protocols using FRE in a ``natural'' way~\cite{naveed15,CCS:DurDubCas16,CCS:PouWri16}, following earlier attacks on keyword search systems~\cite{hore12, arasu13, islam14, cash15}. The  idea is that if the adversary uses public auxiliary information as a prior, it can use Bayesian inference to recover the data.  Naveed~\emph{et al.}~\cite{naveed15} specifically targeted CryptDB~\cite{PRZB11} and showed that in certain applications a large fraction of the attribute values encrypted under DE or OPE/ORE can be recovered.  Their techniques were refined and improved over time~\cite{SP:Paul GSBNR17,EPRINT:BGCRS17}.   I stress that there attacks work no matter what variant or generalization of OPE is used, although for ORE/OPE with non-ideal security the attacks perform better. 
%The response from the authors of CryptDB~\cite{EPRINT:PopZelBal15} claimed that these attacks do not use CryptDB in the applications it was intended for and generated quite a bit of controversy.  My goal is not to take a sides in the debate, but to point out that regardless of whether the attacks use CryptDB in the applications it was intended for, we \emph{do want} secure outsourced database protocols for these applications. 
Moreover, they just rely on access to a copy of the encrypted database (a ``snapshot'' attack).  Grubbs~\emph{et al.}~\cite{Grubbs:2017} argue that  any reasonable adversary would also get access to information about the queries, which corresponds to a stronger security model (a ``persistent'' attack). In~\cite{CCS:KKNO16} I formalize one such model and  show that for range queries in some cases if the adversary sees either the \emph{access pattern} or even the \emph{communication volume} of responses to queries, the encrypted attribute values can be recovered without needing auxiliary data.  Note that my communication volume  attack even works on ODB protocols based in the on SSE-based protocols and protocols built in the natural way from heavyweight cryptographic primitives.  Some recent work improves the attacks~\cite{EPRINT:LacMinPat17}.  

following earlier attacks on keyword search systems~\cite{hore12, arasu13, islam14, cash15}

The search for meaningful notions was initiated much more recently by two of my recent works in submission. First, in~\cite{KKNO17} I~\footnote{I use terms ``I'' and ``my'' in this proposal to refer to my  work, with implicit due credit to my co-authors.} consider
differential privacy~\cite{DMNS06,DKMMN06} of the server's view --- meaning the view of the server doesn't change much if one plaintext is changed --- in a  formalization of the persistent attack model; in fact, I even let the server choose the queries. (One could of course ask that the queries be hidden, which I did not do here but is addressed for keyword search in~\cite{HO18} on attacks, and will be considered in this proposal.)  Second, in~\cite{CLOZZ18} I consider a notion of ``parameter hiding'' in a formalization of the snapshot attack model; this notion roughly means the data has a known shape but its mean and variance are hidden. I do not claim these notions really capture real-world attacks well or give easy-to-understand guarantees but they are at least a step in right direction, and I will combine and build on them substantially in this proposal.


\paragraph*{My Goal.} The ideal outcome of this proposal would be a practical ODB protocol that achieves strong, meaningful security guarantees and resists current and further-developed attacks, based on FRE if possible.  More generally:

 \begin{quote} \emph{The goal of this proposal is to lay a foundation for the design and analysis of ODB protocols, based on FRE when possible, and to build a comprehensive education and outreach plan around it.}
\end{quote}



Toward  this goal there are three main subgoals:
 \begin{itemize}
 \item A foundational but practice-oriented study of FRE.  This comprises coming up with syntax and security notions, and studying relations between them.  The most novel notion is indistinguishability from a random (or even non-uniformly distributed) function preserving some relation, generalizing my approach in~\cite{EC:BCLO09}. %In fact, we still don't have a very good understanding of the leakage of~\cite{EC:BCLO09},  which is in-use in industry.
 In terms of constructions, I think more much attention needs to be paid to polynomial-size message spaces (polynomial in the security parameter) where ciphertext size is independent of domain size; note that all attacks below depend on this (and succeed in a reasonable amount of time).  Additionally, I will  identify new applications of FRE and study  for which functions we can have efficient FRE even in principle. %I also plan to study specific questions concerning ORE  that may to lead practical schemes in some scenarios with \emph{ideal} security, which we can currently only achieve from impractical ``multilinear maps.''\footnote{I refer here to ORE for polynomial-size message spaces with ciphertext length sublinear in the message space size.}
 
 \item Developing a  comprehensive understanding of attacks on ODB protocols. In particular,  \cite{EPRINT:BGCRS17} claims ``optimality'' in the snapshot model but only  counts the expected number of many plaintexts are  recovered.  I will consider recovering some bits of plaintexts or aggregate  statistics.  Additionally, expectation may not be a sufficient versus worst-case performance. On the other hand, the attacks of~\cite{CCS:KKNO16} do not use auxiliary information about the data and I will see if they improved by doing so, extend them to realistic query distributions 
   % that significantly outperforms prior attacks that rely on one or the other. We plan to generalize this attack and also improve it for when access \emph{pattern} is leaked.
 \item Inspired by these attacks, improve on the current protocols meeting  ``meaningful'' security notions .  For bucketization-based protocols, I hope to incorporate the approach of Hacigumus \emph{et al.}~\cite{Hacigumus:2002} into my recent~\cite{KKNO17} to support multiple query types without need to store separate copies of the ``encrypted'' DB.  Second, I  will  to explore ODB protocols meeting based on FRE, hopefully achieving .  The next step is considering stronger guarantees than DP, perhaps combining perhaps security against  attacks in developed in the second subgoal and build protocols achieving.  Particularly, I am interested in building such protocols from the ``ORAM plus sanitizer'' approach of~\cite{KKNO17} by looking at alternative security notions for both components.
 
%I hope to achieve it based on more efficient primitives and synthesize it with my protocols achieving  differential privacy, achieving both guarantees and more simultaneously. %Regarding the ```bucketization'' approach, one issue I hope to resolve is needing to store different copies of the ``encrypted DB'' for each query type, which can be avoided by heuristic protocols~\cite{Hacigumus:2002} . 
 %Regarding the  ``ORAM plus sanitization'' approach, I am interested in variants of the security notions for both the ORAM and sanization; in particular, what notions for ORAM allow one to circumvent the lower bound~\cite{EPRINT:LarNie18}.  %I plan to distinguish between polynomial-size (or in more practical terms, numbers on the order of operations you can perform) vs.~super-polynomial size message space and for the former for which functions we can have secure FRE with ciphertext size non-linear in the size of the message space; this in particular may lead to practical ORE in some scenarios with ideal security. Furthermore, for super-polynomial size message space, I hope to characterize the functions for which we can get secure FRE/FPE from one-way functions (and thus  pseudo-random functions, meaning constructions are likely to be efficient).  Foundationally, a main open question concerning ORE is whether it can be constructed from bilinear maps~\cite{EPRINT:ZanZha17}. 
  \end{itemize} 
  
  

The huge draw of DE and OPE is that they allow ``encryption-oblivious'' query processing, meaning they allow it to be done   \emph{exactly same way} as for unencrypted data. This corresponds what I call function-\emph{preserving} (FPE), which the public algorithm for evaluating the function of the plaintexts from their ciphertexts is simply the function itself.   DE and OPE were incorporated by Popa et al.~\cite{PRZB11} into the well-known CryptDB system~\cite{PRZB11}, which the authors showed could run many SQL queries with modest throughput overhead.  OPE is also in use by companies such as CipherCloud and Skyhigh Networks (acquired by McAfee), as well as many others (see page 2 column 1 of~\cite{FVYSHGSMC17} for a comprehensive list).  One could make a case that the FRE (because it requires minimal changes to the back-end software), and particularly FPE (which requires no changes) approach to ODB protocols has been established as the approach companies and applied researchers prefer.  There are of course alternatives. The original work of Hacigumus \emph{et al.}~\cite{Hacigumus:2002} introduced a ``bucketization'' approach wherein records with close attribute values are encrypted and grouped together, which I make rigorous in my work~\cite{KKNO17}.  %Ironically, as I explain below,  this is so-far the only approach (suitably modified to add noise to each bucket) for which we've been able to prove meaningful security guarantees for for an ODB protocol.  
Work in the cryptographic community also studies structured and symmetric searchable encryption (SSE) --- see Kamara's talk~\cite{kamara15} for an overview --- and approaches based on garbled circuits~\cite{SP:PKVKMCGKB14,SP:FVKKKMB15}, but these are generally more suited for unstructured text and keyword search.

On the other hand, a major issue with DE, OPE, and other forms of FRE (and indeed SSE and garbled circuit based approaches as well protocols as well) are that they are typically proven to have specific leakage, but this says nothing about what can be deduced from this leakage.  After they were developed and incorporated into systems used in practice, the community still lacked a ``meaningful'' security notions for ODB protocols that  gives users a clear, understandable guarantee about their privacy. 
\fi
\iffalse

The search for meaningful notions was initiated much more recently by two of my recent works in submission. First, in~\cite{KKNO17} I~\footnote{I use terms ``I'' and ``my'' in this proposal to refer to my  work, with implicit due credit to my co-authors.} consider
differential privacy~\cite{DMNS06,DKMMN06} of the server's view --- meaning the view of the server doesn't change much if one plaintext is changed --- in a  formalization of the persistent attack model; in fact, I even let the server choose the queries. (One could of course ask that the queries be hidden, which I did not do here but is addressed for keyword search in~\cite{HO18} on attacks, and will be considered in this proposal.)  Second, in~\cite{CLOZZ18} I consider a notion of ``parameter hiding'' in a formalization of the snapshot attack model; this notion roughly means the data has a known shape but its mean and variance are hidden. I do not claim these notions really capture real-world attacks well or give easy-to-understand guarantees but they are at least a step in right direction, and I will combine and build on them substantially in this proposal.
 

 %Interestingly, my constructions of DE and OPE assume an exponential-sized message space, while it is perhaps more practical to assume a polynomial-size message space, which the above attacks assume (\emph{i.e.}, they run in polynomial time in the magnitude of the largest domain element, not the bit-length).

\paragraph*{Meaningful Security Notions.} 
From these attacks, it became clear that we need meaningful security notion for ODB protocols and to achieve them by moving beyond natural  protocols.
%The work of~\cite{EPRINT:LacPat17} defends deterministically encrypted data against ``snapshot'' attacks.
I have two recent work that make some initial progress:
\begin{itemize}
 \item In~\cite{KKNO17} I design ODB protocols that achieve \emph{differential privacy} (DP)~\cite{DMNS06,DKMMN06} of the server's view in the  ``persistent'' attack model of~\cite{CCS:KKNO16}; in fact, we even let the server choose the queries. One could of course ask that the queries be hidden, which I did not do here but is addressed in some of my other work on attacks~\cite{HO18}. (This work is for keyword search on a document corpus, but I want to improve it and extend it to relational DBs.) The  notion I achieve is  \emph{differential privacy} (DP)~\cite{DMNS06,DKMMN06}  of the server's view; its view does not change much if one record is added or deleted. I note that is an interesting question what DP really means in this context and I will study that as part of the proposal. ODB protocols for exact-match and (separately) range queries.   The constructions are  based on the ``bucketization'' approach of~\cite{Hacigumus:2002}, with encrypted dummy records added according to the results of a ``sanitizer''~\cite{STOC:BluLigRot08} (non-interactive differential privacy).  The overhead of these protocols is modest. I also give a more general  construction that combines a sanitizer with oblivious RAM~\cite{Gol87, GO96}, which using off-the-shelf ORAM is less practical (retrieving 1k records takes about 30 seconds).  However, it  (1) supports arbitrary query types and the server stores a single copy of the encrypted DB (2) moves beyond the ``atomic'' model where encrypted records are explicitly stored --- which is as I show in~\cite{KKNO17} is in some sense necessary because for ``one-way attribute'' queries there is no  differentially-private ODB protocol in this model (even if the access pattern includes dummy records), and (3) I believe has the potential to to be significantly make it practical (see below). % Interestingly, my work  inspired the study of relaxed ORAM with differential privacy guarantees~\cite{EPRINT:CCMS17}.
\end{itemize}
Additionally, other (and related, their study being within the scope of the proposal) defenses against snapshot attacks have also been introduced~\cite{EPRINT:PouGriWri17,LP18} that offer different guarantees.
 
\fi 


\iffalse
\paragraph*{ODB protocols.}
 DE supports exact-match queries and OPE/ORE yields ODB protocols supporting range queries in a straightforward manner.
They have been  incorporated into implemented ODB protocols, such as CryptDB \cite{PRZB11}, Cipherbase \cite{arasu13b}, and TrustedDB \cite{BS14}, which combine them with other tricks to support many SQL queries. Regarding the other approaches I consider for designing ODB protocols, my ORAM plus sanitizer approach inspired work on the study of ORAM and other ``oblivious'' algorithms under a differential privacy type notion~\cite{EPRINT:CCMS17}.

\paragraph*{Attacks on ODB protocols.} 
Although various security theorems are proven in work on FRE and ODB protocols, perhaps the best way to access their security is via attacks.
I mostly covered known attacks on ODB protocols built from FRE or in other ways above, but I note that since we know of no practical ORE scheme with ideal leakage the attacks are more sometimes nuanced depending on the scheme considered; I don't try to provide a comprehensive account here.  In~\cite{CCS:KKNO16} I formalize one such model and  show that for range queries in some cases if the adversary sees either the \emph{access pattern} or even the \emph{communication volume} of responses to queries, the encrypted attribute values can be recovered without needing auxiliary data.  Note that my communication volume  attack even works on ODB protocols based in the on SSE-based protocols and protocols built in the natural way from heavyweight cryptographic primitives.  Some recent work improves the attacks~\cite{EPRINT:LacMinPat17}.  
 \fi
 